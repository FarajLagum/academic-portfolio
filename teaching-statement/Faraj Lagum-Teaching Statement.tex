\documentclass[11pt]{article}

\usepackage{epigraph}

\usepackage{color}
\setlength{\oddsidemargin}{0in}
\setlength{\evensidemargin}{0in}
\setlength{\textwidth}{6.5in}
\setlength{\topmargin}{-.5in}
\setlength{\textheight}{10in}
\pagestyle{empty}

%\newcommand{\textblue}[1]{{\leavevmode\color{blue} #1}}
%\newcommand{\textred}[1]{{\leavevmode\color{red} #1}}

\newcommand{\ignore}[1]{}


% define variables

\def\youruniversity{Lambton Collage}
\begin{document}



\begin{center}
{\Large Statement of Teaching Interests} \\[.3in]
{\large Faraj Lagum}\\
faraj.lagum@carleton.ca \\
\vspace*{.5in}
{``\emph{Education is not the learning of facts, but the training of minds to think.}'' \\ ---Albert Einstein.}
\end{center}

%\epigraph{I recall seeing a package to make quotes}{Snowball}

%\vspace*{.5in}






\section{Teaching Experience}
% My interest in teaching and education has started during my postgraduate studies, and my pedagogical experience has evolved over time as I gained more experience from each teaching duty I had. My first teaching experience was as a Teaching Assistant at University of Benghazi during which I was fortunate to practice teaching and learn the best teaching practice of well qualified professors. At that time, I was responsible for leading tutorials, marking assignments, and help the students in their projects. As I enjoy teaching and its associated activities, I used to provide an extensive feedback on the student work within a tight timeline. After three years of being a TA, and based on the based on the positive recommendations of both professors and students I secured a position as a Lecture Assistance (instructor) at the Electrical and electronic engineering department once I finished my MSc. study. 
My passion for teaching and education was ignited during my postgraduate studies when my journey in teaching began as a teaching assistant (TA) at the University of Benghazi. This role allowed me to observe and learn from highly skilled professors. % My responsibilities included leading tutorials, grading assignments, and assisting students with their projects. 
After three years as a TA, and thanks to positive recommendations from professors and students, I secured a position as a Lecture Assistant (instructor) in the Electrical and Electronic Engineering department upon completing my MSc studies.


% my tenure at the University of  lasted more than four years, during which 

I continued my teaching journey at the University of Benghazi from July 2008 to December 2012. During this period, I held a lecturer position in the Electrical and Electronic Engineering department. During tenure at the University of Benghazi, I taught two courses each semester. These courses included Telecommunication Theory I \& II, Electromagnetics I \& II, Microwaves, Antenna \& Propagation, C++, and MATLAB.
Over time, my pedagogical skills have developed and matured through various teaching roles I've undertaken. I thoroughly enjoy teaching and all its related activities. Despite tight deadlines, I used to provide comprehensive feedback on student work. 
In December 2012, I left the University of Benghazi to pursue my PhD in Canada. %This opportunity was made possible by a prestigious scholarship I received from the Libyan Ministry of Higher Education based on a recommendation from the University of Benghazi. 
This marked the next step in my academic journey.




%I continued to teach at University of Benghazi from July 2008 to December  2012 where I left the university to pursue my PhD in Canada after a secured a prestigious scholarship based on the University recommendation form the Libyan Ministry of higher recommendation.  

%In July, 2008, after I finished MSc. study, I secured a position as a Lecture at the Electrical and electronic engineering department, University of Benghazi,  where I continuted to teach for more that four years until left the department and come to Canada to pursue my PhD. 

%During my four years tuner at University of Benghazi I used to teach two courses every semester. These courses include Telecommunication Theory I \& II; Electromagnetics I \& II; Microwaves; and Antenna \& Propagation; C++, and Matlab. 


%At the onset of my teaching career, I primarily focused on clearly explaining the main concepts of the subject and solving various problems in a simplified manner. I initially believed this was all an instructor needed to do in a class. However, after the first semester, I realized that while necessary, this approach was not sufficient for comprehensive student understanding. 

In the early stages of my teaching career, I primarily focused on clearly explaining the main concepts of the subject and solving various problems in a simplified manner. Initially, I believed that this approach sufficed for effective classroom instruction. However, upon reflection after the first semester, I recognized that \ignore{while necessary,} this instructional strategy alone did not adequately foster comprehensive student comprehension and mastery. A more multifaceted approach was essential to achieve deeper understanding among students.
I recognized the importance of developing students' critical thinking skills and problem-solving techniques. I also saw the need to alleviate mathematical anxiety, %especially among undergraduate students, 
rather than solely focusing on solving numerical problems.
To achieve these goals, I actively involved students in the learning process. I encouraged them to ask questions and discuss  practical \ignore{, real-world} examples. Importantly, I worked to unify seemingly disparate concepts, recognizing that many ideas repeated across different modules might initially appear unrelated.
This evolution in my teaching approach has been instrumental in fostering a more engaging and effective learning environment.


%In the beginning of this role, I focused on explaining main concepts and ideas of the subject and on solving various problems to the students clearly  in a simplified manner which thought I this is all an instructor need to do in a class.  Later, after the first semester, I realized it is necessary but not sufficient to make the students fully understand the subject. It realized it is important to develop the students critical thinking capabilities and problem-solving techniques and to reduce the mathematical anxiety, especially to the undergraduate students, rather than focusing on solving numerical problems. To achieve this target, I involved the students in the learning process by asking them questions and encourage them to ask questions, by giving more practical and technical examples from the real world, and more importantly by unifying the seemingly disparate concepts. In particular, I recognized that many concepts that are repeated in several modules may initially appear to be disparate ideas. 



%For instance, in Wireless Communications courses, the wireless receivers employ filters to receive the signals of interest and block other unwanted signals. While in Digital Communications courses, the same filters are used to maximize the signal-to-noise ratio at the receiver. Moreover, in Digital Signal Processing courses, several efficient ways exist to implement such filters. Hence, I like to show the cohesive nature of these filters by presenting their applications in various courses. This unification approach reduces the students’ mathematical anxiety and allows them to focus their effort on the developing of the problem-solving techniques, instead of spending time to understand the same concepts and mathematical analysis over gain, merely because it was presented in a different way or using a different notation. After adopting this teaching practice, the official students’ feedback was positive and encouraging and many students acknowledged this way of teaching to be very efficient and meet their expectations. 

%/newpage


Upon joining Carlton University as a Ph.D. student, I served as a teaching assistant for various courses such as Digital Signal Processing, Digital Communication Theory, Wireless Communications, Operations Research (Optimization), Computer Systems Architecture, Communications Systems Lab, Electronics II, and Computer Networks. This experience illuminated the complexity and diversity inherent in teaching in a multicultural Canadian university. %It encompassed a range of responsibilities, from traditional face-to-face instruction to curriculum development, leveraging digital technology, and crafting engaging tutorials.
Understanding the significance of fostering an inclusive learning environment, I prioritize creating spaces where every student feels esteemed, respected, and equipped to thrive.


%After I joined Carlton university as PhD. student, I used to be a teaching assistance for several courses including Digital Signal Processing, Digital Communication Theory, Wireless Communications, Operations Research (Optimization),  Computer Systems Architecture, Communications Systems Lab, Electronics II, and Computer Networks. I realized that teaching in a multicultural Canadian University is a complex and multifaceted task that includes face-to-face teaching, curriculum design, use of digital technology, and designing engaging tutorial. I recognize the importance of creating an inclusive learning environment where all students feel valued, respected, and empowered to succeed. 

%To achieve this goal, I proactively incorporate diverse perspectives, voices, and experiences into my teaching materials and curriculum. I also strive to create a supportive and inclusive classroom culture where open dialogue, mutual respect, and empathy are encouraged. Furthermore, I actively mentor and support students from underrepresented groups, recognizing the unique challenges they may face and providing them with the necessary resources and support to thrive.

\newpage

During my time at Carleton University, as a Ph.D. student and later as a Postdoctoral Fellow, I provided mentorship to undergraduate and postgraduate students throughout various aspects of their academic pursuits, including course projects, graduation projects, research papers, and Ph.D./master's theses. Through this engagement, my perspective on teaching and learning has evolved significantly. I now perceive students as active partners in the educational process, emphasizing the importance of their active involvement in grasping the global and industrial context of their studies and adequately preparing for the demands of the job market.
Drawing from my industrial experience, I have gained a keen insight into the needs of students as they transition to the job market. This understanding serves as my guiding principle, ensuring that the teaching and learning materials I deliver are aligned with equipping students for success in their future careers.


%At Carleton University, both as PhD. student and later as Postdoctoral Fellow, I also mentored undergrad and post grad students during their course projects, graduation projects, research papers, and master thesis'.  Based on this experience, my view on learning and teaching practice has evolved and now I view students as partners that have to be actively engaged in the teaching and learning activities to better understand the global context of their studies and be prepared to the  job market. Currently, after my my industrial experience, I have a clear understanding of the students’ needs to be prepared to the competitive job market which will be my North Star when it comes to delivering the teaching and learning materials.  %For instance, 

%At Ulster University, I taught three courses with diverse students’ requirements. For instance, in Communications I and II, I conducted a brief survey at the beginning of the semester and I realized that most of the students are working part-time in various telecommunication industries. They came with different expectations that the module somehow should cover topics related to their industry. This is a challenging task as the field of communication is ever-growing and we have a short end-of-term exam to test the students’ knowledge, and hence, the learning outcomes may not properly be achieved. In addition to a final exam, this course has a mid-semester class test that in my view tests the students’ in a way similar to the final exam. I requested to change the coursework for this course to include a research project that covers various telecommunication topics needed by the students. The topics that I suggested for the research project match the recently announced themes of Innovate UK and include 5G communication systems, drone-based communications, applications of artificial intelligence in future networks, and wireless body area networks. The students’ feedback were very positive and encouraging. Another challenging task I faced at Ulster University is teaching large-size classes and design appropriate assessment. I taught Electrical Engineering Fundamentals for first year students with a class size of approximately 175 students.

%After analyzing the poor performance of students in previous years, we realized that the assessment coursework during the semester was not engaging and does not continuously measure the students understanding. That said, I adopted a clicker-based system to test students’ knowledge, at the end of each lecture and seminar, by presenting a simple electrical circuit for the student to analyze and then pick the correct answer. This technique was proven to be effective in engaging the students in the learning process, as well as, encourage them to study for each lecture and seminar. The students’ feedback at the end of the module was very positive and the success rate have improved significantly compared to previous years. However, there were some ripples and a percent of the students were not fully satisfied. This is something that I am keen to improve in future. Also, at Ulster University, I actively participate in the revalidation process of the computing systems program. I proposed and designed a module COM444: Communications and Networks for undergraduate students that covers a range of new wireless technologies. This was an enriching experience for me and I gained a lot about designing modules, choosing suitable assessment type for the students’ cohort, engaging the stakeholders, and balance the coursework over the semester and with respect to other modules.


% \section{Teaching Experience}
%I have more than 14 years of teaching experience. During my career, I have been a trainer, teaching assistant, and lecturer. I had been a full-time lecturer at Electrical and Electronic Engineering Department, University of Benghazi (UoB), Benghazi, Libya, for more than four years. The undergraduate program at UoB has two Electromagnetic courses, one Antenna \& Propagation course, and one Microwaves course. During my working at UoB, I used to teach Electromagnetics II and a Telecommunications courses every term. However, from term to term, I sometimes teach Antenna~\&~Propagation, Electromagnetics I, or Microwaves courses. When I was teaching the electromagnetic courses, I chose the ``\emph{Engineering Electromagnetics}'' by William H. Hayt and ``\emph{Elements of electromagnetics}'' by Matthew N.O. Sadiku are the textbook and the main reference, respectively. Before that, I worked as a TA at UoB for three years during my M.Sc. study where I used to be a TA mostly for the  Circuit theory, Digital Circuits, and Electronics courses. Currently, I am a TA, RA, and Ph.D. candidate at the Systems and Computer Enginerring Departement, Carleton University. Prof. Halim Yanikomeroglu is my supervisor. At Carleton, I have been a TA for several courses. More about my teaching experience is described in my curriculum vitae.
   
   
   
   %Teaching at the University of Benghazi was a transformative experience that significantly shaped my approach to education. Here are a few ways how:

%1. **Exposure to Diverse Courses**: Teaching a variety of courses, from Telecommunication Theory and Electromagnetics to programming languages like C++ and Matlab, broadened my academic perspective. This diversity helped me understand the interconnectedness of different fields and the importance of a holistic approach to engineering education.

%2. **Learning from Experienced Professors**: Working alongside well-qualified professors provided me with invaluable insights into effective teaching methodologies and practices. Observing their pedagogical techniques helped me develop my own unique teaching style.

%3. **Student Interaction**: Leading tutorials and assisting students with their projects allowed me to understand the challenges students face in grasping complex concepts. This experience underscored the importance of patience, clear communication, and practical examples in teaching.

%4. **Feedback and Improvement**: Providing extensive feedback on student work within tight timelines taught me the value of constructive criticism in learning. It also highlighted the need for efficiency and time management in teaching duties.

%5. **Continuous Learning**: The transition from being a Teaching Assistant to a Lecturer necessitated continuous learning and adaptation. This experience instilled in me the belief that as an educator, my learning never stops.

%These experiences at the University of Benghazi have not only shaped my approach to education but also fueled my passion for teaching and continuous learning. They have made me a better educator and a lifelong learner.



\section{Teaching Philosophy}
% My Experience framed my  teaching philosophy which, in conclusion, is centered on creating a dynamic, inclusive, and student-centered learning environment that fosters intellectual curiosity, critical thinking, and academic excellence. 
Shaped by my experiences, my teaching philosophy revolves around establishing a dynamic, inclusive, and student-centered learning environment that nurtures intellectual curiosity, encourages critical thinking, and promotes academic excellence.
For my point of view, teaching is the act of communicating knowledge and coaching minds to think critically. Teaching is about stretching the brains, not just filling them with information. As Plutarch (45 AD---120 AD) said: ``the mind is not a vessel that needs filling, but wood that needs igniting.'' To make the teaching intellectually rewarding for the students, I believe that the instructor must set high standards for the course and maintain motivating his students to adopt the deep learning approach. To make that happen, I believe that the instructor should create a non-threatening and fun learning atmosphere in which the students can explore new concepts, make mistakes, and get feedback without any judgment. Moreover, the instructor should also pose stimulating and intriguing questions that are very challenging for the students. In my opinion, teaching is about training the students to analyze and interpret the meaning of each formula or text. 


\section{Teaching Interests}

% I am well trained to teach courses such as the Electromagnetic Waves. I have profound experience in teaching similar courses. During my graduate and undergraduate studies, I attended several courses in Electromagnetics, Antenna, and Microwaves, where I always get excellent marks. My M.Sc. was about computational electromagnetics. The title of my thesis is: ``\emph{Moment Method Analysis of a Liner Wire Antenna Using Exponential Subdomain Function.}'' Indeed, Electromagnetic Waves is among the fields of my expertise. I am interested in teaching the Electromagnetic Waves course. I am also interested in teaching the Switching Circuits course.


%I already prepared to teach courses on the following subjects: Artificial Intelligence and Machine Learning, Data structures and Algorithms, Computer Networks,  Software Engineering courses, Mathematics (Calculus, Linear Algebra, Probability, and Optimization), digital signal processing, signals and systems, Cloud Computing.  I am also open to prepare to teach other subjects are required by the department.
%My understanding of, and practice in, effective learning, teaching, and assessment and feedback have been well developed. I am looking forward to continuing my commitment to excellent teaching at Carleton University. 

I am prepared to teach a range of courses in Computer Science with focus on Artificial Intelligence and Machine Learning, including Deep Learning, Foundation Models, Natural Language Processing (NLP), Large Language Models (LLMs), Computer Vision, Data (text, audio, and video) Analysis, Data Structures and Algorithms, Computer Networks, Software Engineering, Data Science, Big Data, Mathematics (covering Calculus, Linear Algebra, Probability, and Optimization), Digital Signal Processing, Signals and Systems, and Cloud Computing. I am also open to teaching other subjects required by the department. Additionally, I am open to developing courses that are in demand within the industry but may not typically be part of standard computer science programs, such as Shell Scripting, Command-line Environment (Linux), Version Control (Git) and System Design.
My expertise extends to effective learning methodologies, teaching strategies, and assessment and feedback practices. I am committed to delivering high-quality instruction and fostering an environment conducive to student success at \youruniversity. %I am eager to further expand my teaching repertoire and contribute to the academic growth of our students.




%I already prepared to teach courses on the following subjects: digital signal processing, digital communications, communication theory, wireless communications, stochastic process, signals and systems, and computer networks. 

%I am also open to prepare to teach other subjects are required by the department. My commitment for being an excellent teacher in higher education is sincere. I am an Associate Fellow of the Higher Education Academy and currently enrolled in the course of post-graduatePage  certificate in higher education practice (PgCHEP) that I expect to finish by the end of this year and hold the full fellowship. My understanding of, and practice in, effective learning, teaching, and assessment and feedback have been well developed. I am looking forward to continuing my commitment to excellent teaching at University of Saskatchewan












\end{document}
