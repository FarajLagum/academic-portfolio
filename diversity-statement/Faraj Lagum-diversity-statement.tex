\documentclass[11pt]{article}

\usepackage{epigraph}

\usepackage{color}
\setlength{\oddsidemargin}{0in}
\setlength{\evensidemargin}{0in}
\setlength{\textwidth}{6.5in}
\setlength{\topmargin}{-.5in}
\setlength{\textheight}{10in}
\pagestyle{empty}

%\newcommand{\textblue}[1]{{\leavevmode\color{blue} #1}}
%\newcommand{\textred}[1]{{\leavevmode\color{red} #1}}

\newcommand{\ignore}[1]{}

\begin{document}



\begin{center}
{\Large Diversity Statement} \\[.3in]
{\large Faraj Lagum}\\
faraj.lagum@carleton.ca \\

\ignore
{
\vspace*{.5in}
{``\emph{Education is not the learning of facts, but the training of minds to think.}'' \\ ---Albert Einstein.}
}

\end{center}

%\epigraph{I recall seeing a package to make quotes}{Snowball}

%\vspace*{.5in}





%My journey toward championing diversity and inclusion has been deeply influenced by my experiences transitioning from being a majority member to feeling marginalized upon arriving in Canada. Growing up, I was part of a homogeneous country, where I enjoyed the privilege of familiarity with the language, culture, and societal norms. However, when I immigrated to Canada, I suddenly found myself navigating unfamiliar territory, carrying the burden of assimilation in the new culture, and facing daunting challenges daily. This transition opened my eyes to the experiences of marginalized individuals facing similar challenges. It was a humbling experience that made me acutely aware of the barriers individuals from underrepresented groups often encounter. My experience has shaped my perspective on the importance of diversity and inclusion in academia.

%As I pursued higher education and embarked on my academic career, this experience fueled my passion for fostering diversity and inclusion in academia. 
%I realized that true academic excellence can only be achieved when individuals from all backgrounds are given equal opportunities to thrive and contribute their unique perspectives.
%In my previous roles as a Teaching Assistant and Postdoctoral Researcher at Carleton University, I had the opportunity to mentor students from various backgrounds. This experience reinforced my belief in the importance of diversity in academia, and I am committed to continuing these efforts as a professor.

My personal experiences have profoundly influenced my understanding and advocacy for diversity. Growing up in a homogeneous country, I enjoyed the privilege of familiarity with the language, culture, and societal norms. However, upon immigrating to Canada, I found myself navigating unfamiliar territory, carrying the burden of assimilation, and facing cultural understanding challenges. This transition from being a majority member in a homogeneous country to feeling marginalized was a humbling experience that made me acutely aware of the barriers often encountered by visible minorities.



This experience fueled my passion for fostering diversity and inclusion in academia as I pursued higher education and embarked on my academic career. As a Teaching Assistant and Postdoctoral Researcher at Carleton University, I was privileged to mentor students from diverse backgrounds, including visible minorities. This role allowed me to explore new avenues where inclusivity and equity could be emphasized within my professional role, particularly for visible minorities.


%As an academic professional, I am deeply committed to fostering a diverse and inclusive learning environment. I believe that diversity in all its forms enriches the educational experience and drives innovation and creativity. In my view, an inclusive environment is a safe environment where all culturally and racially diverse populations are inspired to embrace their identity and flourish in their academic journey without fear of judgment or social rejection. I am dedicated to dismantling barriers and creating pathways for individuals from visible minority groups to succeed in academia. I believe that by embracing diversity and fostering an inclusive environment, we can unlock the full potential of every student and pave the way for a brighter and more equitable future.

I am deeply committed to fostering a diverse and inclusive learning environment as an academic professional. I believe that diversity in all its forms enriches the educational experience and drives innovation and creativity. I am dedicated to dismantling barriers and creating pathways for individuals from visible minority groups to succeed in academia. 

If allowed to serve as a professor at your institution, I will work to promote diversity and inclusion in all aspects of academic life.

%If allowed to serve as a professor at your institution, I will work to promote diversity and inclusion in all aspects of academic life. In my classrooms and research groups, I will strive to create an inclusive environment where every student feels valued and respected, regardless of their background. %Drawing upon my own experiences of being a member of a visible minority group, I will work to provide support and guidance to students who may be facing similar challenges. 
%Furthermore, I am committed to leveraging my position as an academic to advocate for systemic changes that promote diversity and inclusion within Carleton University. I emphasize the importance of fostering an inclusive educational environment where all educational institution members are treated equally. 



%I am excited about the possibility of contributing to your institution’s commitment to diversity, and I look forward to the opportunity to discuss my qualifications further.











\end{document}
