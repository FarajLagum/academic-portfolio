\documentclass[11pt]{article}

\usepackage{epigraph}

\usepackage{color}
\setlength{\oddsidemargin}{0in}
\setlength{\evensidemargin}{0in}
\setlength{\textwidth}{6.5in}
\setlength{\topmargin}{-.5in}
\setlength{\textheight}{9in}
\pagestyle{empty}

%\newcommand{\textblue}[1]{{\leavevmode\color{blue} #1}}
%\newcommand{\textred}[1]{{\leavevmode\color{red} #1}}

\begin{document}



\begin{center}
{\Large Statement of Teaching Interests} \\[.3in]
{\large Faraj Lagum}\\
\vspace*{.5in}
{``\emph{Education is not the learning of facts, but the training of minds to think.}'' \\ ---Albert Einstein.}
\end{center}

%\epigraph{I recall seeing a package to make quotes}{Snowball}

%\vspace*{.5in}
\section{Teaching Philosophy}
For my point of view, teaching is the act of communicating knowledge and coaching minds to think critically. Teaching is about stretching the brains, not just filling them with information. As Plutarch (45 AD---120 AD) said: ``the mind is not a vessel that needs filling, but wood that needs igniting.'' To make the teaching intellectually rewarding for the students, I believe that the instructor must set high standards for the course and maintain motivating his students to adopt the deep learning approach. To make that happen, I believe that the instructor should create a non-threatening and fun learning atmosphere in which the students can explore new concepts, make mistakes, and get feedback without any judgment. Moreover, the instructor should also pose stimulating and intriguing questions that are very challenging for the students. In my opinion, teaching is about training the students to analyze and interpret the meaning of each formula or text. 



\section{Teaching Experience}
I have more than 12 years of teaching experience. During my career, I have served as a trainer, teaching assistant, and lecturer. I had been a full-time lecturer at the Electrical and Electronic Engineering Department, University of Benghazi (UoB), Benghazi, Libya, for more than four years. The undergraduate program at UoB has two Electromagnetic courses, one Antenna \& Propagation course, and one Microwaves course. During my work at UoB, I used to teach Electromagnetics II and Telecommunications courses every term. Further, from term to term, I sometimes teach Antenna~\&~Propagation, Electromagnetics I, or Microwaves courses. When I was teaching the electromagnetic courses, I chose the ``\emph{Engineering Electromagnetics}'' by William H. Hayt and ``\emph{Elements of electromagnetics}'' by Matthew N.O. Sadiku are the textbook and the main reference, respectively. The average class size is around 55 students with a small variance in the number of the students.
Before that, I worked as a TA at UoB for three years during my M.Sc. study where I used to be a TA mostly for the Circuit theory, Digital Circuits, and Electronics courses. 



During my Ph.D. study in the Systems and Computer Engineering Department at Carleton University, Ottawa, ON, I was a TA for the following courses: SYSC 4405 Digital Signal Processing (Fall 2018, Winter 2018); SYSC 4604 Digital Communication Theory (Fall 2018); SYSC 4607 Wireless Communications (Winter 2018); SYSC3200 Industrial Engineering (Optimization) (Fall 2017, Fall 2016, and Fall 2015); SYSC 4310 Computer Systems Architecture (Winter 2017); SYSC 4701 Communications Systems Lab (Winter 2016); ELEC 3509 Electronics II (Summer 2015). The class sizes vary widely from 12 students to 120 students.


%\section{Teaching Interests}

%Without any exaggeration, I am well trained to teach courses such as the Electromagnetic Waves. I have profound experience in teaching similar courses. During my graduate and undergraduate studies, I attended several courses in Electromagnetics, Antenna, and Microwaves, where I always get excellent marks. My M.Sc. was about computational electromagnetics. The title of my thesis is: ``\emph{Moment Method Analysis of a Liner Wire Antenna Using Exponential Subdomain Function.}'' Indeed, Electromagnetic Waves is among the fields of my expertise. I am interested in teaching the Electromagnetic Waves course. I am also interested in teaching the Switching Circuits course.





   


\end{document}
